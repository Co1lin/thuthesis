% !TeX root = ../thuthesis-example.tex

\chapter{引言}

\section{研究背景}
% 1 page;
% deep learning
近年来,深度学习技术在计算机视觉、自然语言处理等领域取得了显著成果,普通大众频繁使用智能创作、聊天机器人等基于深度神经网络模型的各种智能服务正在成为现实,深度学习系统效率与安全的重要性日益凸显。

% computation graph & deep learning compilers
深度学习编译器是深度学习系统(或称深度学习框架)中的重要组成部分,是将训练完毕的深度学习模型部署到生产环境进行服务的重要工具。如传统编译器在将源代码编译为可执行文件的过程中可以进行常数折叠、死代码消除、循环展开等优化,深度学习编译器可以在将高级语言描述的深度神经网络转换为不同硬件平台对应的底层机器代码的过程中,对深度学习模型的抽象数据结构表示——计算图,做类似的一系列优化,以减少模型的资源占用或提高模型的推理速度。

然而,在部署模型的过程中使用编译器可能带来可靠性风险,因为编译器中计算图构建、图优化、机器代码发射等组件的具体实现中可能存在漏洞。尽管开发者遵循软件工程规范为大量接口实现了单元测试,但手工编写的这些测试丰富度有限,不能尽可能多地暴露深度学习编译器中的全部风险。

% fuzz testing techniques
% fuzz test for deep learning compilers
模糊测试技术是软件工程领域中常用的自动化测试方法。它通过生成并向程序输入大量随机数据,同时监控其运行状态与返回结果,来寻找程序中的漏洞,最早被用于对 Unix 工具的测试。
由于模糊测试可以免除手工构造测例的繁重人力劳动,因此多年以来被用于充分测试各种软件系统以提高它们的可靠性,如浏览器、图形渲染引擎、车载软件系统等。近年来,随着深度学习的兴起,研究者开始探索如何对深度学习系统,包括深度学习编译器,进行模糊测试。


\section{研究挑战}
% 1 pages
% valid models as inputs
模糊测试在早期被用于测试 Unix 工具时,仅仅通过生成随机字符串的方式构造测例。然而,这种简单的测例生成算法并不适用于编译器等复杂系统的测试。这是因为,复杂系统的输入往往是遵循某种语法规则的结构化数据,而不合法的输入会导致程序在刚开始解析输入时便报错退出,于是只能测试程序的输入处理部分,而无法触发后续的程序主体。

因此,对深度学习编译器进行模糊测试的首要挑战是生成合法的神经网络。神经网络的合法性可以被分解为两部分:每个算子自身的合法性与算子之间连接的合法性。前者指对于每个算子,其输入,包括张量与其它属性参数,它们的组合应当符合算子本身的定义;后者指相邻算子之间,前一算子的输出张量需要符合后一算子对输入张量的要求,主要是考虑张量的形状。二者共同确保了整个神经网络模型的合法性。

% diverse operator/tensor/attr types
而为了实现充分测试,在模型合法的基础上应尽可能提高模型多样性,从而触发深度学习编译器中尽可能多的优化逻辑。多样性的不同维度及各自的意义至少包含以下几点:(1)算子类型的多样性:深度学习编译器对不同算子、不同的相邻算子组合,有不同的优化实现;(2)张量形状的多样性:不同的输入张量形状对应的最优计算策略不同(如并行策略中的块大小);(3)属性参数的多样性:很多算子接受非张量输入作为属性参数控制计算行为,如卷积算子中的步长,而不同的属性参数可能对应不同的优化策略。

然而,在合法性约束下,模型多样性与测例生成方法的自动化程度存在矛盾。简单的测例生成方法具有高度的自动化程度,但难以处理合法性约束条件复杂而各不相同的多样算子。引入人工定义的规则可以一一处理多样的算子,但大大降低了生成方法的自动化程度;或需要耗费大量人力,牺牲一定的实用性;或难以自动高效地生成大量测例,降低了整体的多样性。总之,所生成测例的合法性与多样性,以及测试框架的自动化程度,共同对深度学习编译器的模糊测试构成了挑战。而现有工作均不能很好地在三者之间寻求平衡,因此存在较大改进空间。

\section{本文工作}
% 1 paragraph
针对上述问题与挑战,首先,接下来的第 2 章将介绍相关工作,并分析各自的权衡取舍与得失,为后文提出改进策略提供参考基础。其次,第 3 、 4 章将分别介绍一种新的神经网络模糊测试用例生成方法的主要组成部分,分别是基于插桩的具体算子数据集的构建,与混合算子计算图的生成。然后,第 5 章介绍本工作中的一项工程性贡献,即图结构中间表示与可读代码的相互转换,它可以暴露更多漏洞、提高测试有效性、并丰富测例来源。最后,第 6 章对测试过程中所寻找的典型漏洞进行展示,并做简要的根因分析,可帮助直观理解深度学习编译器中的实现漏洞,与启发高效调试等其它相关子领域的研究。

\iffalse

近期最新工作 NNSmith 采用完全符号化规则约束下的计算图生成算法解决了合法性问题,然而其在效率、多样性与测例真实性等方面都仍然存在不足:
\begin{enumerate}
    \item 不能自动为深度学习库支持的全部成百上千种算子定义测例生成所需的计算规则与约束;
    \item 在生成测例模型时只利用了很少一部分算子,算子多样性与覆盖率低;
    \item 生成测例时没有参考单元测试与开源项目中现有的算子调用方法,因此所生成测例与现实中人工编写的代码存在一定距离。
\end{enumerate}

\chapter{论文主要部分的写法}

研究生学位论文撰写,除表达形式上需要符合一定的格式要求外,内容方面上也要遵循一些共性原则。

通常研究生学位论文只能有一个主题(不能是几块工作拼凑在一起),该主题应针对某学科领域中的一个具体问题展开深入、系统的研究,并得出有价值的研究结论。
学位论文的研究主题切忌过大,例如,“中国国有企业改制问题研究”这样的研究主题过大,因为“国企改制”涉及的问题范围太广,很难在一本研究生学位论文中完全研究透彻。



\section{论文的语言及表述}

除国际研究生外,学位论文一律须用汉语书写。
学位论文应当用规范汉字进行撰写,除古汉语研究中涉及的古文字和参考文献中引用的外文文献之外,均采用简体汉字撰写。

国际研究生一般应以中文或英文书写学位论文,格式要求同上。
论文须用中文封面。

研究生学位论文是学术作品,因此其表述要严谨简明,重点突出,专业常识应简写或不写,做到立论正确、数据可靠、说明透彻、推理严谨、文字凝练、层次分明,避免使用文学性质的或带感情色彩的非学术性语言。

论文中如出现一个非通用性的新名词、新术语或新概念,需随即解释清楚。



\section{论文题目的写法}

论文题目应简明扼要地反映论文工作的主要内容,力求精炼、准确,切忌笼统。
论文题目是对研究对象的准确、具体描述,一般要在一定程度上体现研究结论,因此,论文题目不仅应告诉读者这本论文研究了什么问题,更要告诉读者这个研究得出的结论。
例如:“在事实与虚构之间:梅乐、卡彭特、沃尔夫的新闻观”就比“三个美国作家的新闻观研究”更专业、更准确。



\section{摘要的写法}

论文摘要是对论文研究内容的高度概括,应具有独立性和自含性,即应是 一篇简短但意义完整的文章。
通过阅读论文摘要,读者应该能够对论文的研究 方法及结论有一个整体性的了解,因此摘要的写法应力求精确简明。
论文摘要 应包括对问题及研究目的的描述、对使用的方法和研究过程进行的简要介绍、 对研究结论的高度凝练等,重点是结果和结论。

论文摘要切忌写成全文的提纲,尤其要避免“第 1 章……;第 2 章……;……”这样的陈述方式。



\section{引言的写法}

一篇学位论文的引言大致包含如下几个部分:
1、问题的提出;
2、选题背 景及意义;
3、文献综述;
4、研究方法;
5、论文结构安排。
\begin{itemize}
  \item 问题的提出:要清晰地阐述所要研究的问题“是什么”。
    \footnote{选题时切记要有“问题意识”,不要选不是问题的问题来研究。}
  \item 选题背景及意义:论述清楚为什么选择这个题目来研究,即阐述该研究对学科发展的贡献、对国计民生的理论与现实意义等。
  \item 文献综述:对本研究主题范围内的文献进行详尽的综合述评,“述”的同时一定要有“评”,指出现有研究状态,仍存在哪些尚待解决的问题,讲出自己的研究有哪些探索性内容。
  \item 研究方法:讲清论文所使用的学术研究方法。
  \item 论文结构安排:介绍本论文的写作结构安排。
\end{itemize}



\section{正文的写法}

本部分是论文作者的研究内容,不能将他人研究成果不加区分地掺和进来。
已经在引言的文献综述部分讲过的内容,这里不需要再重复。
各章之间要存在有机联系,符合逻辑顺序。



\section{结论的写法}

结论是对论文主要研究结果、论点的提炼与概括,应精炼、准确、完整,使读者看后能全面了解论文的意义、目的和工作内容。
结论是最终的、总体的结论,不是正文各章小结的简单重复。
结论应包括论文的核心观点,主要阐述作者的创造性工作及所取得的研究成果在本领域中的地位、作用和意义,交代研究工作的局限,提出未来工作的意见或建议。
同时,要严格区分自己取得的成果与指导教师及他人的学术成果。

在评价自己的研究工作成果时,要实事求是,除非有足够的证据表明自己的研究是“首次”、“领先”、“填补空白”的,否则应避免使用这些或类似词语。

\fi