\input{regression-test.tex}
\documentclass[degree=doctor]{thuthesis}

\begin{document}
\START


\frontmatter
\setcounter{page}{3}
\showoutput
\listoftables
\clearpage
\OMIT


\mainmatter
\chapter{引言}


\chapter{文献综述}

\clearpage
\setcounter{page}{13}
\begin{table}
  \centering
  \caption{城市排水系统规划的三个层次及对应的具体工作}
\end{table}

\clearpage
\setcounter{page}{25}
\begin{table}
  \centering
  \caption{国内关于排水管道定线和水力计算方面研究的学位论文}
\end{table}

\clearpage
\setcounter{page}{26}
\begin{table}
  \centering
  \caption{用于“面+网+厂”空间设计的主要模型及其功能}
\end{table}

\clearpage
\setcounter{page}{28}
\begin{table}
  \centering
  \caption{不确定性因素对城市污水数量水质影响的定性分析}
\end{table}

\clearpage
\setcounter{page}{35}
\begin{table}
  \centering
  \caption{SRES 和 RCPs 温室气体排放情景}
\end{table}

\clearpage
\setcounter{page}{36}
\begin{table}
  \centering
  \caption{动力降尺度与统计降尺度的对比}
\end{table}

\begin{table}
  \centering
  \caption{考虑气候变化和城镇化(用地、人口)对城市排水系统设计影响的相关研究}
\end{table}


\chapter{不确定条件下分流制城市排水系统优化设计方法研究}

\clearpage
\setcounter{page}{45}
\begin{table}
  \centering
  \caption{城市规划地块(UB)与系统设计单元(DU)之间的对应关系}
\end{table}

\clearpage
\setcounter{page}{46}
\begin{table}
  \centering
  \caption{基准设计条件的基础数据收集及其分类}
\end{table}

\clearpage
\setcounter{page}{47}
\begin{table}
  \centering
  \caption{可行系统设计过程中的基础数据使用过程}
\end{table}

\clearpage
\setcounter{page}{48}
\begin{table}
  \centering
  \caption{参数集成优化约束条件中对应的基础数据支持}
\end{table}

\clearpage
\setcounter{page}{51}
\begin{table}
  \centering
  \caption{不确定性影响因素的定性分析}
\end{table}

\clearpage
\setcounter{page}{55}
\begin{table}
  \centering
  \caption{系统设计中雨水排放口与污水处理厂的空间位置选择原则}
\end{table}

\clearpage
\setcounter{page}{60}
\begin{table}
  \centering
  \caption{不确定情景下雨水系统性能评估指标及计算方法}
\end{table}

\clearpage
\setcounter{page}{62}
\begin{table}
  \centering
  \caption{不确定情景下污水系统性能评估指标设计}
\end{table}


\chapter{含有不确定性参数的城市排水系统优化设计模型}

\clearpage
\setcounter{page}{68}
\begin{table}
  \centering
  \caption{UDS Model 中设计单元(DU)的对应排水行为}
\end{table}

\clearpage
\setcounter{page}{70}
\begin{table}
  \centering
  \caption{不同设计类型 DU 的径流系数(Runoff\_Pra)取值}
\end{table}

\clearpage
\setcounter{page}{72}
\begin{table}
  \centering
  \caption{不同类型 DU 对再生水(RWW)的需求信息}
\end{table}

\clearpage
\setcounter{page}{76}
\begin{table}
  \centering
  \caption{成本计算公式中的参数取值}
\end{table}

\clearpage
\setcounter{page}{78}
\begin{table}
  \centering
  \caption{UDS Model 中雨水系统和污水系统设计的输入数据}
\end{table}

\clearpage
\setcounter{page}{85}
\begin{table}
  \centering
  \caption{满流排水管道水力约束条件矩阵(Hydra\_st\_F)}
\end{table}

\clearpage
\setcounter{page}{89}
\begin{table}
  \centering
  \caption{污水系统模型中的约束条件}
\end{table}

\clearpage
\setcounter{page}{90}
\begin{table}
  \centering
  \caption{非满流排水管道水力约束条件矩阵(Hydra\_st\_NF)}
\end{table}

\clearpage
\setcounter{page}{96}
\begin{table}
  \centering
  \caption{SWMM 模型 input 文件信息}
\end{table}


\chapter{案例研究:昆明市城北片区排水系统设计}

\clearpage
\setcounter{page}{110}
\begin{table}
  \centering
  \caption{研究区域分流制排水系统设计目标}
\end{table}

\clearpage
\setcounter{page}{111}
\begin{table}
  \centering
  \caption{城北片区概化的 DU 个数及面积信息}
\end{table}

\clearpage
\setcounter{page}{112}
\begin{table}
  \centering
  \caption{用于形成基准设计情景的数据收集}
\end{table}

\begin{table}
  \centering
  \caption{不同类型 DU 的污染物平均浓度(EMC)取值}
\end{table}

\clearpage
\setcounter{page}{113}
\begin{table}
  \centering
  \caption{用于形成不确定性评估情景集合的数据和制备}
\end{table}

\clearpage
\setcounter{page}{114}
\begin{table}
  \centering
  \caption{不同温室气体排放情景下研究区域降雨量变化}
\end{table}

\begin{table}
  \centering
  \caption{降雨雨峰系数变化的情景设置}
\end{table}

\begin{table}
  \centering
  \caption{城镇化带来的土地利用变化情景设置}
\end{table}

\clearpage
\setcounter{page}{115}
\begin{table}
  \centering
  \caption{人口总量增长的情景设计取值}
\end{table}

\begin{table}
  \centering
  \caption{不同节水器具普及情景下人均日用水量与基准情景用量的比值}
\end{table}

\begin{table}
  \centering
  \caption{基于气候相似性分析的城市居民生活用水量情景设置}
\end{table}

\clearpage
\setcounter{page}{119}
\begin{table}
  \centering
  \caption{研究区域内污水处理厂可行选址的基本信息}
\end{table}

\begin{table}
  \centering
  \caption{基准条件下满足研究区域排水需求的污水厂设计组合}
\end{table}

\clearpage
\setcounter{page}{122}
\begin{table}
  \centering
  \caption{第 17 号方案中第 10 区(17\#\_SubR10)管道水力参数一览表}
\end{table}

\clearpage
\setcounter{page}{123}
\begin{table}
  \centering
  \caption{第 17 号方案中第 10 区(17\#\_SubR10)泵站参数设置一览表}
\end{table}

\begin{table}
  \centering
  \caption{典型设计方案的性能对比}
\end{table}

\clearpage
\setcounter{page}{127}
\begin{table}
  \centering
  \caption{NDSP 排序后属于第一级的系统设计方案信息}
\end{table}

\clearpage
\setcounter{page}{132}
\begin{table}
  \centering
  \caption{Flow increase 情景中关键变化比例与水力性能改变的对应关系统计}
\end{table}

\clearpage
\setcounter{page}{136}
\begin{table}
  \centering
  \caption{符合设计目标要求的雨水系统设计方案个数、目标值和方案设计信息}
\end{table}

\clearpage
\setcounter{page}{138}
\begin{table}
  \centering
  \caption{研究区域污水系统设计推荐方案}
\end{table}


\chapter{不确定条件下城市排水系统设计规律识别与分析}

\clearpage
\setcounter{page}{144}
\begin{table}
  \centering
  \caption{采用不同 SO 数量与管道重现期进行系统设计的方案个数及性能区间}
\end{table}

\clearpage
\setcounter{page}{149}
\begin{table}
  \centering
  \caption{城北片区不同设计排水量情景下的可行 WWTP 组合数目}
\end{table}

\clearpage
\setcounter{page}{150}
\begin{table}
  \centering
  \caption{不同情景下非支配排序中处于第一级的系统设计方案变化情况}
\end{table}


\OMIT
\end{document}
