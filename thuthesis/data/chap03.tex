% !TeX root = ../thuthesis-example.tex

\chapter{数学符号和公式}

\section{数学符号}

中文论文的数学符号默认遵循 GB/T 3102.11—1993《物理科学和技术中使用的数学符号》
\footnote{原 GB 3102.11—1993,自 2017 年 3 月 23 日起,该标准转为推荐性标准。}。
该标准参照采纳 ISO 31-11:1992 \footnote{目前已更新为 ISO 80000-2:2019。},
但是与 \TeX{} 默认的美国数学学会(AMS)的符号习惯有所区别。
具体地来说主要有以下差异:
\begin{enumerate}
  \item 大写希腊字母默认为斜体,如
    \begin{equation*}
      \Gamma \Delta \Theta \Lambda \Xi \Pi \Sigma \Upsilon \Phi \Psi \Omega.
    \end{equation*}
    注意有限增量符号 $\increment$ 固定使用正体,模板提供了 \cs{increment} 命令。
  \item 小于等于号和大于等于号使用倾斜的字形 $\le$、$\ge$。
  \item 积分号使用正体,比如 $\int$、$\oint$。
  \item
    偏微分符号 $\partial$ 使用正体。
  \item
    省略号 \cs{dots} 按照中文的习惯固定居中,比如
    \begin{equation*}
      1, 2, \dots, n \quad 1 + 2 + \dots + n.
    \end{equation*}
  \item
    实部 $\Re$ 和虚部 $\Im$ 的字体使用罗马体。
\end{enumerate}

以上数学符号样式的差异可以在模板中统一设置。
另外国标还有一些与 AMS 不同的符号使用习惯,需要用户在写作时进行处理:
\begin{enumerate}
  \item 数学常数和特殊函数名用正体,如
    \begin{equation*}
      \uppi = 3.14\dots; \quad
      \symup{i}^2 = -1; \quad
      \symup{e} = \lim_{n \to \infty} \left( 1 + \frac{1}{n} \right)^n.
    \end{equation*}
  \item 微分号使用正体,比如 $\dif y / \dif x$。
  \item 向量、矩阵和张量用粗斜体(\cs{symbf}),如 $\symbf{x}$、$\symbf{\Sigma}$、$\symbfsf{T}$。
  \item 自然对数用 $\ln x$ 不用 $\log x$。
\end{enumerate}


英文论文的数学符号使用 \TeX{} 默认的样式。
如果有必要,也可以通过设置 \verb|math-style| 选择数学符号样式。

关于量和单位推荐使用
\href{http://mirrors.ctan.org/macros/latex/contrib/siunitx/siunitx.pdf}{\pkg{siunitx}}
宏包,
可以方便地处理希腊字母以及数字与单位之间的空白,
比如:
\SI{6.4e6}{m},
\SI{9}{\micro\meter},
\si{kg.m.s^{-1}},
\SIrange{10}{20}{\degreeCelsius}。



\section{数学公式}

数学公式可以使用 \env{equation} 和 \env{equation*} 环境。
注意数学公式的引用应前后带括号,通常使用 \cs{eqref} 命令,比如式\eqref{eq:example}。
\begin{equation}
  \frac{1}{2 \uppi \symup{i}} \int_\gamma f = \sum_{k=1}^m n(\gamma; a_k) \mathscr{R}(f; a_k).
  \label{eq:example}
\end{equation}

多行公式尽可能在“=”处对齐,推荐使用 \env{align} 环境。
\begin{align}
  a & = b + c + d + e \\
    & = f + g
\end{align}



\section{数学定理}

定理环境的格式可以使用 \pkg{amsthm} 或者 \pkg{ntheorem} 宏包配置。
用户在导言区载入这两者之一后,模板会自动配置 \env{thoerem}、\env{proof} 等环境。

\begin{theorem}[Lindeberg--Lévy 中心极限定理]
  设随机变量 $X_1, X_2, \dots, X_n$ 独立同分布, 且具有期望 $\mu$ 和有限的方差 $\sigma^2 \ne 0$,
  记 $\bar{X}_n = \frac{1}{n} \sum_{i+1}^n X_i$,则
  \begin{equation}
    \lim_{n \to \infty} P \left(\frac{\sqrt{n} \left( \bar{X}_n - \mu \right)}{\sigma} \le z \right) = \Phi(z),
  \end{equation}
  其中 $\Phi(z)$ 是标准正态分布的分布函数。
\end{theorem}
\begin{proof}
  Trivial.
\end{proof}

同时模板还提供了 \env{assumption}、\env{definition}、\env{proposition}、
\env{lemma}、\env{theorem}、\env{axiom}、\env{corollary}、\env{exercise}、
\env{example}、\env{remar}、\env{problem}、\env{conjecture} 这些相关的环境。
